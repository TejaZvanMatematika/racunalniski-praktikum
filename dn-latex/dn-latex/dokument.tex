\documentclass[11pt]{article}
\usepackage[T1]{fontenc}
\usepackage[utf8]{inputenc}
\usepackage[a4paper, margin=2.5cm]{geometry}
\usepackage[slovene]{babel} 
\usepackage{graphicx}
\usepackage{amsthm}   % definicije okolij za izreke, definicije, ...
\usepackage{hyperref}
\usepackage{makeidx}

\makeindex

% Definicija okolij izrek, posledica
{\theoremstyle{plain}
\newtheorem{izrek}{Izrek}
\newtheorem{posledica}[izrek]{Posledica}
}

% Definicija okoli za definicije in vaje
{\theoremstyle{definition}
\newtheorem{definicija}[izrek]{Definicija}
\newtheorem{vaja}[izrek]{Vaja}}

\newcommand{\F}{\mathcal{F}}

\title{Brownovo gibanje}
\author{Matej Rojec}
\date{}

\begin{document}

\maketitle

Brownovo gibanje (več v \cite{karatzas1991brownian}) je intuitivno slučajen proces, % Sklic na knjigo
ki predstavlja naključno gibanje delcev v mediju.

\begin{figure}[h!]
  \centering
  \includegraphics[width=0.4\textwidth]{PerrinPlot2.pdf}
  \caption{Reprodukcija slike iz Jean Baptiste Perrin,
  \emph{Mouvement brownien et réalité moléculaire},
  Ann. de Chimie et de Physique (VIII) 18, 5--114, 1909}
\end{figure}
    % Slika: PerrinPlot2.pdf
    % Napis pod sliko: 
    % Reprodukcija slike iz Jean Baptiste Perrin, \emph{Mouvement brownien et réalité moléculaire}, Ann. de Chimie et de Physique (VIII) 18, 5-114, 1909

    % Začetek definicije
    \begin{definicija}
        Standardno Brownovo gibanje $\{B_t\}_{t \geq 0}$ je slučajen proces z naslednjimi lastnostmi: 
        \begin{enumerate}
        \item $B_0 = 0$.
        \item Prirastki $B_{t_n} - B_{t_{n-1}}, B_{t_{n-1}} - B_{t_{n-2}}, \ldots, B_2 - B_1, B_1 - B_0$ so neodvisne slučajne spremenljivke, za vsak $t_0 \leq t_1 \leq \cdots \leq t_{n-1} \leq t_n$.
        \item Za vsak $t \geq 0$ in $h > 0$ velja $B_{t+h} - B_t \sim \mathcal{N}(0, h)$.
        \item Funkcija $t \mapsto B_t$ je zvezna skoraj gotovo.
        \end{enumerate}
        

    \end{definicija}
    % Konec definicije
    
    Preden zapišemo izrek, definirajmo še pojem časa ustavljanja.
    
    % Začetek definicije
    \begin{definicija}
        Slučajna spremenljivka $\tau$ na verjetnostnem prostoru \((\Omega, \F, \texttt{P})\) z vrednostmi v 
        je čas ustavljanja glede na filtracijo ??, če velja ??.
    \end{definicija}
    % Konec definicije
    
    Zdaj lahko zapišemo izrek \ref{thm:stopped_brownian}. % Sklic na izrek z oznako thm:stopped_brownian
    
    % Začetek izreka
    \begin{izrek}
    \label{thm:stopped_brownian}
    Naj bo $\{B_t\}_{t \geq 0}$?? (standardno) Brownovo gibanje, ?? čas ustavljanja glede na 
    ?? in naj bo ??.
    Potem je tudi proces:
    \[
    \hat{B} := \{B_{T+t} - B_T \mid t \geq 0\}
    \]
    (standardno) Brownovo gibanje in neodvisen od $\F_T$.
    \end{izrek}
    % Konec izreka
    
\bibliographystyle{plain}
\bibliography{knjiga}

\end{document}